\section{Simulation Analysis}
\label{sec:simulation}

In this section of the lab assignment, the main goal was to improve the amplifier circuit, based on the given script. For that, it was important to understand the importance that some key components have in the output. One of the most important element of the gain stage it the transistor. Its purpose is to increase the gain. In order to produce satisfactory results, it is important that it is working in Forward Active Region. This is ensured by choosing appropriate values for $V_{cc}$ voltage source and resistor $R_1$ and $R_2$.
\par
In order to decrease the effect of an eventual 0V DC component in the input voltage source, it is important to have an input coupling capacitor.
\par
We also used a bypass capacitor with the objective of attenuating the effects of $R_E$, reducing the circuit's impedance for high frequencies. This happens because, for higher frequencies, a capacitor can be aproximated to a short circuit, so the current will flow through instead of flowing exclusively through the resistor. By adding this element, we can reduce the thermal effect without compromising the total impedance for higher frequencies.
\par
It is also important to notice that the function of $R_C$ is to regulate the output signal. However, we need to be careful while doing so, because we can significantly decrease the gain.
\par
For last, the output stage's function is to decrease as much as possible the output impedance, while maintaining a gain of approximately 1.
\par
With all this said, in the table below we can see the total input and output impedances of the circuit

\begin{table}[H]
  \centering
  \begin{tabular}{|l|r|}
    \hline    
    {\bf Variable} & {\bf Value [$\Omega$]} \\ \hline
    \input{../sim/zin_tab.tex}
    \input{../sim/zout_tab.tex}
  \end{tabular}
  \caption{Input and output impedances.}
  \label{tab:sim1}
\end{table}

As we wanted, the input impeadance is high and the output impeadance is low. However, even though the output impedance is lower than the resistance of the load, we consider it to be higher than desirable.
\par
We can now analyze the frequency response of the circuit, as seen in the image below

\begin{figure}[H] \centering
\includegraphics[width=0.7\linewidth]{../sim/gain.pdf}
\caption{Frequency Response.}
\label{fig:sim_deviation}
\end{figure}

It was also plotted a transient analysis for f = 1kHz, 

\begin{figure}[H] \centering
\includegraphics[width=0.7\linewidth]{../sim/vout.pdf}
\caption{Trasient analysis for f = 1 kHz.}
\label{fig:sim_deviation}
\end{figure}

By observing the transient analysis, we can see that there are no major losses of the original signal, as the output signal was still a sine wave. The initial oscillations correspond to the transitory period.
\par
By analysing the frequency response, we can get very important informations about how the circuit behaves, like its lower and higher cutoff frequencies, its bandwidth and its gain. We can see all of the obtained results in the table below

\begin{table}[H]
  \centering
  \begin{tabular}{|l|r|}
    \hline    
    {\bf Variable} & {\bf Value [$\Omega$]} \\ \hline
    \input{../sim/sim_tab.tex}
  \end{tabular}
  \caption{Input and output impedances.}
  \label{tab:sim1}
\end{table}

We can finally calculate the total cost of the circuit to obtain the merit figure of this circuit. Both these values are presented in the table below

\begin{table}[H]
  \centering
  \begin{tabular}{|l|r|}
    \hline    
    {\bf Variable} & {\bf Value [$\Omega$]} \\ \hline
    \input{../sim/merit_tab.tex}
  \end{tabular}
  \caption{Input and output impedances.}
  \label{tab:sim1}
\end{table}

Comparing these results with those obtained in the theorethical analysis, we can conclude that both differ a bit. In the theorethical ones, the final mean voltage wasn't a perfect 12V in contrast with the ones obtained in the simulation, and the ripple was also higher. This was to be expected, since \textit{Ngspice} uses a diode model much closer to the reality comparing to the ones used in theorethical analysis, since did aproximations in order to obtain simpler equations.






