\section{Simulation Analysis}
\label{sec:simulation}

The main goal of this section was to improve the bandpass filter circuit for a specific central frequency, and a specific gain in dB. In order to find these values, we plotted the frequency response of the circuit, as we can see in the figure below

\begin{figure}[H] \centering
\includegraphics[width=0.7\linewidth]{../sim/gain.pdf}
\caption{Frequency Response.}
\end{figure}

The obtained values for these quantities is shown in the table below

\begin{table}[H]
  \centering
  \begin{tabular}{|l|r|}
    \hline    
    {\bf Variable} & {\bf Value[Hz or dB]} \\ \hline
    \input{../sim/simval_tab}
  \end{tabular}
  \caption{Obtained frequences and gain.}
\end{table}

This results were similar to the ones obtained in the theorethical analysis, but not exactly the same. This is due to the differences in the Op-Amp model used in the two analysis.
\par

We can also see the frequency response in the figure below, but this time for the phase in degrees

\begin{figure}[H] \centering
\includegraphics[width=0.7\linewidth]{../sim/phase.pdf}
\caption{Frequency Response.}
\end{figure}

This is one of the biggest differences when compared to the results obtained in the theorethical analysis. This can can be explained by the fact that \textit{Ngspice} maintains its angles in the interval from -180º to 180º, and its Op-Amp model adds two more poles, which is not considered on the theorethical analysis. 
\par
In the table below, we can see the total input and output impedances of the circuit. We want to have a high input impedance and a low output impedance. 

\begin{table}[H]
  \centering
  \begin{tabular}{|l|r|}
    \hline    
    {\bf Variable} & {\bf Value [$\Omega$]} \\ \hline
    \input{../sim/zin_tab}
    \input{../sim/zout_tab}
  \end{tabular}
  \caption{Input and output impedances.}
\end{table}

As we wanted, we achived a high input impedance. However, the outout impedance wasn´t very low, due to the resistor in paralel with the Op-Amp.
\par
Finally, the gain and central frequency aswell as the circuit's cost and the total merit are presented in the table below

\begin{table}[H]
  \centering
  \begin{tabular}{|l|r|}
    \hline    
    {\bf Variable} & {\bf Value]} \\ \hline
    \input{../sim/merit_tab}
  \end{tabular}
  \caption{Cost and merit.}
\end{table}



