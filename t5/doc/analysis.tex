\section{Theoretical Analysis}
\label{sec:analysis}
In this section we will use incremental analysis to analyse the circuit. The circuit is composed of three stages, first a high pass filter, then an amplification stage (with the OP-AMP) and finally a low pass filter.
\par

\subsection{Frequency response}
Firstly we use the node method on node $v_+$, where the current passing in the OP-AMP is 0:

\begin{equation}\label{eq:v_+}
v_+=\frac{\frac{1}{Z_1}}{\frac{1}{Z_1}+\frac{1}{R_1}} v_i = \frac{1}{1+\frac{Z_1}{R_1}} v_i\\
\end{equation}

Next the non-inverting amplifier OP-AMP circuit gives us two equations:

\begin{equation}\label{eq:OP-AMP}
\begin{cases}
(1)  v_-=v_+\\
(2)  v_o=(1+\frac{R_3}{R_2})v_+\\
\end{cases}
\end{equation}

Again using the node method on $V_{out}$ we get:

\begin{equation}\label{eq:v_out}
v_{out}=\frac{\frac{1}{R_4}}{\frac{1}{R_4}+\frac{1}{Z_2}} v_o = \frac{1}{1+\frac{R_4}{Z_2}} v_o\\
\end{equation}

Finally, combining equations \ref{eq:v_+}, \ref{eq:OP-AMP}.2 and \ref{eq:v_out}, we get:

\begin{equation}\label{eq:v_final}
v_{out}=\frac{(1+\frac{R_3}{R_2})}{(1+\frac{R_4}{Z_2})(1+\frac{Z_1}{R_1})} v_i\\
\end{equation}

Using this equation we can get the frequency response, both for the gain and the phase. Below we show both graphs.
\par

\begin{figure}[H] \centering
\includegraphics[width=0.7\linewidth]{../mat/gain.pdf}
\caption{Frequency Response (Gain)}
\label{fig:freq}
\end{figure}

\par

\begin{figure}[H] \centering
\includegraphics[width=0.7\linewidth]{../mat/phase.pdf}
\caption{Frequency Response (Phase)}
\label{fig:freq}
\end{figure}


\subsection{Central frequency}
To calculate the central frequency, we will use the geometric mean between the lower cut off frequency and upper cut off frequency:

\begin{equation}\label{eq:whatever}
\begin{cases}
Lf=\frac{1}{2 \pi R_1 C_1}\\
Uf=\frac{1}{2 \pi R_4 C_2}\\
Cf=\sqrt{Lf \times Uf}\\
\end{cases}
\end{equation}

For the output impedance, $v_i=0$ makes both $v_+$ and $v_-$ zero. This makes it so that $Z_{out}=Z_2//(R_4 + R_2//Z_{OA})$. $Z_{OA}$ is zero so a parallel with it is also zero which gives the final equation:

\begin{equation}\label{eq:z_o}
Z_{out}=\frac{1}{\frac{1}{Z_2} + \frac{1}{R_4}}\\
\end{equation}

For the input impedance, we have $Z_i= Z_1+(R_1//Z_{OA+})$. But $Z_{OA+}=\infty$ so it becomes:

\begin{equation}\label{eq:z_o}
Z_i=Z_1 + R_1\\
\end{equation}

Next we present a table with the values for the central frequency:


\begin{table}[H]
  \centering
  \begin{tabular}{|l|r|}
    \hline    
    {\bf Variable} & {\bf Value [Hz, dB or $\Omega$]} \\ \hline
    \input{../mat/op1_tab.tex}
  \end{tabular}
  \caption{Obtained frequencies, gain and impedances}
  \label{tab:sim1}
\end{table}

%=========================================================================================================================


