\newpage

\section{Simulation Analysis}
\label{sec:simulation}

For the circuit simulation we used the \textit{Ngspice} software. It is important to point that, to simulate a perfect transformer, we needed to replace it by one linear current dependent current source and one linear voltage dependent voltage source. In Figure \ref{fig:sim_output} we can se the voltage after the envelope detector (red) and the voltage at the end of the circuit (blue).

\begin{figure}[H] \centering
\includegraphics[width=0.7\linewidth]{../sim/vout.pdf}
\caption{Output of the Envelope Detector (red) and Voltage Regulator (blue).}
\label{fig:sim_output}
\end{figure}

In Figure \ref{fig:sim_deviation} we have the graph of the voltage at the output subtracted by 12, so we can see better the fluctuations that occur in the output voltage around 12V.

\begin{figure}[H] \centering
\includegraphics[width=0.7\linewidth]{../sim/deviation.pdf}
\caption{Deviation from perfect 12V output.}
\label{fig:sim_deviation}
\end{figure}

As we can see, a stable 12V output voltage with some deviatons was obtained. In Table \ref{tab:sim_values} we can see the average output voltage (avgout) aswell as its ripple and the final merit.

\begin{table}[H]
  \centering
  \begin{tabular}{|l|r|}
    \hline    
    {\bf Variable} & {\bf Value [V or A]} \\ \hline
    \input{../sim/M_tab}
  \end{tabular}
  \caption{Average value ate the output (V), ripple (V) and merit obtained.}
  \label{tab:sim1}
\end{table}

Comparing these results with those obtained in the theorethical analysis, we can conclude that both differ a bit. In the theorethical ones, the final mean voltage wasn't a perfect 12V in contrast with the ones obtained in the simulation, and the ripple was also higher. This was to be expected, since \textit{Ngspice} uses a diode model much closer to the reality comparing to the ones used in theorethical analysis, since did aproximations in order to obtain simpler equations.






