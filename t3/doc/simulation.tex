\newpage

\section{Simulation Analysis}
\label{sec:simulation}

For the circuit simulation we used the \textit{Ngspice} software. It is important to point that, to simulate a perfect transformer, we needed to replace it by one linear current dependent current source and one linear voltage dependent voltage source. In Figure ** we can se the voltage after the envelope detector and the voltage at the end of the circuit.

%\begin{figure}[H] \centering
%\includegraphics[width=0.7\linewidth]{../sim/vn.pdf}
%\caption{Simulated Natural Solution.}
%\label{fig:sim-graph3}
%\end{figure}

In Figure ** we have the graph of the voltage at the output subtracted by 12, so we can see better the fluctuations that occur in the output voltage around 12V.

%\begin{figure}[H] \centering
%\includegraphics[width=0.7\linewidth]{../sim/vn.pdf}
%\caption{Simulated Natural Solution.}
%\label{fig:sim-graph3}
%\end{figure}

As we can see, a stable 12V output voltage with some deviatons was obtained. In Table ** we can see the average output voltage aswell as its ripple.

%\begin{table}[H]
  %\centering
  %\begin{tabular}{|l|r|}
    %\hline    
    %{\bf Variable} & {\bf Value [V or A]} \\ \hline
   % \input{../sim/OP_tab}
  %\end{tabular}
  %\caption{Node voltage and branch current obtained by the simulation analysis.}
 % \label{tab:sim1}
%\end{table}

Finally, with these results, we obtained a merit of ...






