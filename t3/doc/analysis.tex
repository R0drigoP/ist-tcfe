\section{Theoretical Analysis}
\label{sec:analysis}

\subsection{Full Wave Bridge Rectifier}
For the theoretical analysis of the circuit we simplify the rectifier as a voltage source as shown in the circuit below:

IMAGEM

We considered the ideal diode + voltage source model for the diodes, so the voltage $V_s$ is given by $V_s= |V_1|-2\times V_{on}$.
\par
However the diodes make it so that it will work as open circuit when the voltage in the capacitor, $V_c$, is superior to the voltage in said voltage source, $V_s$.

\subsection{Envelope}
Since we know that the voltage on the terminals of the envelope is given by either $V_s$ or $V_c$, we need to know in which intervals it has each each value. The circuit will have a cyclic behaviour, with half the period of the original voltage source (because of the full wave rectifier), which is $T=0.01$.
\par
Firstly we need to find the instant in which the voltage passes from $V_s$ to $V_c$, and we will call it $t_{off}$. This happens when $V_s$ is an open circuit, so we use the nodal analysis on node 1 to get:

\begin{equation}\label{eq:t_off}
\frac{V_s}{R_1}+\frac{V_s+V_2}{R_2}+C\times\frac{dV_s}{dt}=0
\end{equation}

Where we consider $V_2=11.7$ and $V_s=|\frac{230}{10} sin(wt)|-2\times V_{on}$ is a function of time. Solving this equation using octave, we obtained $V_{off}=0.0050 + 0.01k$ (k integer).
\par
Then we use the value of $t_{off}$ in the first cycle to find the instant where the voltage goes back to $V_s$, $t_{on}$. This happens when $V_s=V_c$ again:

\begin{equation}\label{eq:t_on}
V_s=V_s(t_{off})\times e^{-\frac{t-t{off}}{RC}}
\end{equation}

Where R is the equivalent resistance as seen in the capacitor and is given by $R=\frac{R_1 R_2}{R_1 + R_2}$. Again solving with octave, $t_{on}=0.014714 + 0.01k$ (k integer). It is important to note that the solving method can give either this solution, or the trivial solution $t_{on}=t_{off}$.
\par
Now that we have the intervals we can plot a graph of the voltage in the envelope, which is shown later in graph \ref{fig:tensaoF}.

\subsection{Voltage regulator}
Consider the voltage in the envelope as $V1=V_1 + v_1$ in which $V_1$ is the continuous component (given by the average, $\overline{V1}$) and $v_1$ is the oscillation (given by $\overline{V1} - V1$).
Doing the same for V2, we already know that $V_2$ is 11.7, so we only need to calculate $v_2$ using the incremental analysis and node analysis methods.

\begin{equation}\label{eq:v_2}
\begin{cases}
v_2=v_1\times \frac{18 r_d}{18 r_d + R_2} \\
r_d=\frac{V_{on}}{V_T e^{\frac{V_{on}}{V_T}}}\\
\end{cases}
\end{equation}

Where $v_1= V1 - V_1$ and $r_d=12.7727$ is the diode's internal resistance.
\par
Finally we can plot the output voltage of our circuit, $V2=V_2 + v_2$. The graph below shows the plot of both the envelope voltage and the final output voltage:

\begin{figure}[H] \centering
\includegraphics[width=0.7\linewidth]{../mat/tensao.pdf}
\caption{Envelope and Output Voltages}
\label{fig:tensaoF}
\end{figure}

For a better analysis of the output voltage the next plot shows the difference between the the output voltage and the pretended voltage, $V2-12$:

\begin{figure}[H] \centering
\includegraphics[width=0.7\linewidth]{../mat/oscilaçao.pdf}
\caption{Deviation of the Output Voltage}
\label{fig:desvio}
\end{figure}

%=========================================================================================================================


