\section{Theoretical Analysis}
\label{sec:analysis}

\subsection{Circuit for $t<0$}
Using Kirchhoff Current Law (KCL) at all nodes, we end up with these equations:
\begin{equation}\label{eq:node}
  \begin{cases}
    (1) V_1 = V_s \\
    (2) \frac{V_1 - V_2}{R_1} + \frac{V_3 - V_2}{R_2} + \frac{V_5 - V_2}{R_3} = 0 \\
    (3) \frac{V_2 - V_3}{R_2} + K_b(V_2 - V_5) = 0 \\
    (5) V_5 - V_8 + K_d\frac{V_7}{R_6} = 0 \\
    (6) \frac{V_5 - V_6}{R_5} - K_b(V_2-V_5) = I_c \\
    (7) \frac{V_8 - V_7}{R_7} - \frac{V_7}{R_6} = 0 \\
    (8) \frac{V_7 - V_8}{R_7} - \frac{V_5}{R_4} + \frac{V_6 - V_5}{R_5} + \frac{V_2 - V_5}{R_3} = -I_c \\
    I_c = 0V \\
  \end{cases}
\end{equation}
Where $V_i$ represents the voltage at node $i$. It is important to notice that, because nodes 1 and 4 were connected to a voltage source, $V_s$, we use the source equation on equation \ref{eq:node}.1. Similarly for nodes 5 and 8, which are connected to $V_d$, we use the source equation \ref{eq:node}.5. Finally, for equation \ref{eq:node}.7, we consider the super-node 5-8. We consider that the circuit has already reached equilibrium, so the current in the capacitor, $I_c$, is zero.
\par
simplifying these expressions and substituting variables by their numeric number given in table \ref{tab:values}, and solving this system of linear equations using \textit{Octave}, we end up with:
\begin{table}[H]
  \centering
  \begin{tabular}{|l|r|}
    \hline    
    {\bf Variable} & {\bf Value [A or V]} \\ \hline
    \input{../mat/op1_tab}
  \end{tabular}
  \caption{Node voltage and brach current obtained by the theorethical analysis.}
  \label{tab:node}
\end{table}
\par

\subsection{Equivalent Resistance}
\label{subsec:Req}
To calculate the equivalent resistance, $R_{eq}$, as seen in the capacitor's terminals, we replace it with a voltage source $V_x=V_6-V_8$ and remove all the independent sources (in our case, make $V_s=0$). We also simplified the circuit by replacing $R_6$ and $R_7$ by $R_{67}=R_6+R_7$. We need this procedure, since there isn't other way to compute $R_{eq}$ as seen from the capacitor's terminals. We then run nodal analysis to get the following equations:
\begin{equation}\label{eq:nodeReq}
\begin{cases}
  (2) \frac{V_1 - V_2}{R_1} + \frac{V_3 - V_2}{R_2} + \frac{V_5 - V_2}{R_3} = 0 \\
  (3) \frac{V_2 - V_3}{R_2} + K_b(V_2 - V_5) = 0 \\
  (5) V_5 - V_8 + K_d\frac{V_8}{R_{67}} = 0 \\
  (6) V_6 - V_8 = Vx \\
  (8) \frac{V_8}{R_{67}} - \frac{V_5}{R_4} + \frac{V_2 - V_5}{R_3} - K_b(V_2-V_5)= 0 \\
\end{cases}
\end{equation}

Again, because nodes 6 and 8 were connected to a voltage source, $V_x$, we use the source equation on equation \ref{eq:nodeReq}.6, and for equation \ref{eq:nodeReq}.8, we consider the super-node 5-6-8. Equation \ref{eq:nodeReq}.5 is still the source equation for nodes 5 and 8.
After solving the system, to find the current passing through $V_x$, we can rewrite equation in node 6 as $I_x= \frac{R_5 - R_6}{R_5} - K_b(V_2 - V_5)$, and then get $R_{eq}=\frac{V_x}{I_x}$.
The results obtained are:

\begin{table}[H]
  \centering
  \begin{tabular}{|l|r|}
    \hline
        {\bf Variable} & {\bf Value} \\ \hline
        I'_A & 0.000280737730604\\ hline
I'_B & -0.000294385330347\\ hline
I'_C & 0.000969408567879\\ hline
I'_D & -0.001038309112650\\ hline

  \end{tabular}
  \caption{Equivalent Resistance.}
  \label{tab:mesh}
\end{table}


\subsection{Time Reponse}
In this section we will find the evolution of the circuit in respect to time, with frequency $f= 1kHz$.
Firstly the natural solution of $v_6(t)$ is a negative exponential given by $v_{6n}=V_xe^{-\frac{t}{R_{eq}C}}$, as shown in the graph:

\begin{figure}[H] \centering
\includegraphics[width=0.7\linewidth]{../mat/natural.pdf}
\caption{Theorethical Natural Response.}
\label{fig:sim_op}
\end{figure}


To find the forced solution we will run nodal analysis using a phasor $\widetilde{V_s}=1$ V and computing C with its impedance $Z_c=\frac{1}{jwC}$. We get the equation system:

\begin{equation}\label{eq:nodeForced}
  \begin{cases}
  (1) V_1 = 1 \\
  (2) \frac{V_1 - V_2}{R_1} + \frac{V_3 - V_2}{R_2} + \frac{V_5 - V_2}{R_3} = 0 \\
  (3) \frac{V_2 - V_3}{R_2} + K_b(V_2 - V_5) = 0 \\
  (5) V_5 - V_8 + K_d\frac{V_7}{R_6} = 0 \\
  (6) \frac{V_5 - V_6}{R_5} - K_b(V_2-V_5) + jwC(V_8-V_6)= 0 \\
  (7) \frac{V_8 - V_7}{R_7} - \frac{V_7}{R_6} = 0 \\
  (8) \frac{V_7 - V_8}{R_7} - \frac{V_5}{R_4} + \frac{V_6 - V_5}{R_5} + \frac{V_2 - V_5}{R_3} - jwC(V_8-V_6) = 0 \\
\end{cases}
\end{equation}

The complex amplitudes of the nodes obtained by solving the system is shown in the table below:

\begin{table}[H]
  \centering
  \begin{tabular}{|l|r|}
    \hline
        {\bf Phasor} & {\bf Value (V)} \\ \hline
        \input{../mat/op2_tab}
  \end{tabular}
  \caption{Complex amplitudes (Phasors).}
  \label{tab:comp}
\end{table}
\par

The forced solution $v_6f(t)= |\widetilde{V_6}|sin(2\pi ft + \angle \widetilde{V_6})$

Now we get the total solution by summing the natural and forced solutions.
The graph below shows the comparison between $v_6(t)$ and $v_s(t)$ from -5ms to 20ms.

\begin{figure}[H] \centering
\includegraphics[width=0.7\linewidth]{../mat/total.pdf}
\caption{Theorethical Total Solution.}
\label{fig:sim_op}
\end{figure}


\subsection{Frequency Response}
Since the only component that is affected by frequency is the capacitor, the next graph shows both the magnitude of $v_s$, $v_6$ and $v_c=v_6 - v_8$ in decibels, as a function of frequency. 

\begin{figure}[H] \centering
\includegraphics[width=0.7\linewidth]{../mat/magnitude.pdf}
\caption{Theorethical Frequency Response: Magnitude.}
\label{fig:sim_op}
\end{figure}

Finally, the next graph shows the phase in degrees of $v_s$, $v_6$ and $v_c$ as a function of frequency.

\begin{figure}[H] \centering
\includegraphics[width=0.7\linewidth]{../mat/phase.pdf}
\caption{Theorethical Frequency Response: Phase.}
\label{fig:sim_op}
\end{figure}
%=========================================================================================================================


