\section{Simulation Analysis}
\label{sec:simulation}

\subsection{Node Analysis}
\label{subsec:node_sim}

Table \ref{tab:node_sim} shows the node voltages obtained using the simulated operating point results for the circuit under analysis
\begin{table}[H]
  \centering
  \begin{tabular}{|l|r|}
    \hline    
    {\bf Node number} & {\bf Voltage (V)} \\ \hline
    \input{../sim/V_tab}
  \end{tabular}
  \caption{Operating point. A variable preceded by @ is of type {\em current}
    and expressed in Ampere; other variables are of type {\it voltage} and expressed in
    Volt.}
  \label{tab:node_sim}
\end{table}
Compared to the theoretical analysis results, we notice that both are similar, having small numeric differences between them (~ 0.2\%). The most likely cause of this is due to the fact that \textit{Octave} and \textit{Ngspice} have different numerical precision.

\subsection{Mesh Analysis}

Table \ref{tab:mesh_sim} shows the current flowing in each component using the simulated operating point results for the circuit under analysis
\begin{table}[H]
  \centering
  \begin{tabular}{|l|r|}
    \hline    
    {\bf Component} & {\bf Current (A)} \\ \hline
    \input{../sim/I_tab}
  \end{tabular}
  \caption{Operating point. A variable preceded by @ is of type {\em current}
    and expressed in Ampere; other variables are of type {\it voltage} and expressed in
    Volt.}
  \label{tab:mesh_sim}
\end{table}
**Table**
Compared to the theoretical analysis results, we notice that, like in subsection \ref{subsec:node_sim}, both are similar, having differences ~5\%. We think the causes are the same described earlier, with the adition that \textit{Ngspice} uses node analysis in it's simulations, which causes an increased deviation of these values, compared to those in subsection \ref{subsec:node_sim}.


