\section{Simulation Analysis}
\label{sec:simulation}

For the circuit simulation we used the \textit{Ngspice} software. It is important to point that, for this analysis we needed to add an independent voltage source with value 0V to compute the current $I_c$, needed on the dependent voltage source, $V_c$. Figure \ref{fig:cir_sim} represents the simulated circuit

\begin{figure}[H] \centering
\includegraphics[width=0.7\linewidth]{cir_sim.pdf}
\caption{Simulated circuit.}
\label{fig:cir_sim}
\end{figure}

\subsection{Node Analysis}
\label{subsec:node_sim}

Table \ref{tab:node_sim} shows the node voltages obtained using the simulated operating point results for the circuit under analysis
\begin{table}[H]
  \centering
  \begin{tabular}{|l|r|}
    \hline    
    {\bf Node number} & {\bf Voltage (V)} \\ \hline
    \input{../sim/V_tab}
  \end{tabular}
  \caption{Node voltage obtained by the simulation analysis.}
  \label{tab:node_sim}
\end{table}
Compared to the theoretical analysis results, we notice that both are similar, having small numeric differences between them ($\sim$ 0.2\%). The most likely cause of this is due to the fact that \textit{Octave} and \textit{Ngspice} have different numerical precision.

\subsection{Mesh Analysis}

Table \ref{tab:mesh_sim} shows the current flowing in each component using the simulated operating point results for the circuit under analysis
\begin{table}[H]
  \centering
  \begin{tabular}{|l|r|}
    \hline    
    {\bf Component} & {\bf Current (A)} \\ \hline
    \input{../sim/I_tab}
  \end{tabular}
  \caption{Simulated Current flowing through each component.}
  \label{tab:mesh_sim}
\end{table}

Compared to the theoretical analysis results, we notice that, like in subsection \ref{subsec:node_sim}, both are similar, having differences $\sim$ $1\times10^{-4}$\%. We think the causes are the same that we described earlier.


