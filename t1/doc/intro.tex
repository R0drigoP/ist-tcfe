\section{Introduction}
\label{sec:introduction}

% state the learning objective
The objective of this laboratory assignment is to study a circuit containing resistors that follow Ohm's law, independent and linearly dependent voltage and current sources. With this analysis, we want to determine the potential at each of the eight nodes and the current flowing through all of the ten branches.
With this purpose, we'll use both node method, that is, the usage of KVL theorem at all the nodes, considering the voltage of the nodes, and mesh method, that is, applying KCL theorem in all of the four meshes, each one with a current defined by us. (De mais??)
\par
The circuit explained, as well as all numbered nodes and mesh currents, can be seen in Figure \ref{fig:cir_intro}

\begin{figure}[H] \centering
\includegraphics[width=0.7\linewidth]{cir_intro.pdf}
\caption{Analyzed circuit.}
\label{fig:cir_intro}
\end{figure}


In Section \ref{sec:analysis}, a theoretical analysis of the circuit is
presented. In Section \ref{sec:simulation}, the circuit is analysed by
simulation, and the results are compared to the theoretical results obtained in
Section \ref{sec:analysis}. The conclusions of this study are outlined in
Section \ref{sec:conclusion}. (De menos? dizer que vamos uma continuação do que dissemos no inicio na parte teórica)
