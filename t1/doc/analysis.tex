\section{Theoretical Analysis}
\label{sec:analysis}

In this section, the circuit shown earlier in Figure \ref{fig:circuit} is analyzed, using both node and mesh methods, as explained in section \ref{introduction}. It is important to notice that node 0 corresponds to ground, meaning we assign it has having null potential.
\par
%A table with all the given values can be seen in table \ref{tab:value}
**tabela com os valores

\subsection{Node method}
Using Kirchhoff Current Law (KCL) at all nodes, we end up with these equations:
\begin{equation}\label{eq:node}
  \begin{cases}
  (0) V_0 = 0V \\
  (1) I_d = \frac{V_1 - V_6}{R_5} + I_b \\
  (2) I_b = \frac{V_2 - V_3}{R_2} \\
  (3) \frac{V_2 - V_3}{R_2} = \frac{V_3 - V_4}{R_1} + \frac{V_b}{R_3} \\
  (4) V_4 - V_5 = V_a \\
  (5) \frac{V_6 - V_5}{R_4} = \frac{V_5 - V_7}{R_6} + \frac{V_4 - V_3}{R_1} \\ %Dizer que esta ultima parcela é Ir4->r3?
  (6) V_6 = V_c \\
  (7) \frac{V_5 - V_7}{R_6} = \frac{V_7}{R_7} \\
  I_b = K_bV_b = K_b(V_3 - V_6) \\
  V_c = K_cI_c = K_c \frac{V_5 - V_7}{R_6} \\
\end{cases}
\end{equation}
Where $V_i$ represents the voltage at node $i$. It is important to notice that, because nodes 4 and 5 were connected to a voltage source, $V_a$, we use the source equation on equation \ref{eq:node}.4, and for equation \ref{eq:node}.5, we consider the super-node 4-5. We used a similar though process for equation \ref{eq:node}.6.
\par
Substituting variables by their numeric number given in table \ref{tab:node}, and solving this system of linear equations using \textit{Octave}, we end up with:
\begin{table}[H]
  \centering
  \begin{tabular}{|l|r|}
    \hline    
    {\bf Node number} & {\bf Voltage (V)} \\ \hline
    \input{../mat/op1_tab}
  \end{tabular}
  \caption{Node voltage obtained by the theorethical analysis.}
  \label{tab:node}
\end{table}
\par
We can get the current flowing through each branch using Ohm's law.

\subsection{Mesh method}
Using Kirchhoff Voltage Law (KVL) in all meshes, we can write these equations:
\begin{equation}\label{eq:mesh}
\begin{cases}
  (A) R_1I'_A + R_3(I'_A + I'_B) + R_4(I'_A + I'_C) = V_a \\
  (B) I'_B = I_b = K_bV_b = K_bR_3(I'_A + I'_B) \\
  (C) R_4(I'_A+I'_C) + R_6I'_C + R_7I'_C = K_cI_c = K_cI'_C \\
  (D) I'_D = -I_d \\
\end{cases}
\end{equation}
Because mesh B was connected to a current source, we can write the equation \ref{eq:mesh}.B for this mesh. Similarly, we can write equation \ref{eq:mesh}.D, for mesh D. As done in the previous subsection, we can replace the variables by their numeric number given in table \ref{tab:mesh}. Solving this system of linear equations using \textit{Octave}, we end up with:
\begin{table}[H]
  \centering
  \begin{tabular}{|l|r|}
    \hline
        {\bf Mesh name} & {\bf Current (A)} \\ \hline
        I'_A & 0.000280737730604\\ hline
I'_B & -0.000294385330347\\ hline
I'_C & 0.000969408567879\\ hline
I'_D & -0.001038309112650\\ hline

  \end{tabular}
  \caption{Theorethical Current flowing in each mesh.}
  \label{tab:mesh}
\end{table}
Using this information, we can now solve for the current flowing in each component, as seen in table \ref{tab:comp}
\begin{table}[H]
  \centering
  \begin{tabular}{|l|r|}
    \hline
        {\bf Component} & {\bf Current (A)} \\ \hline
        \input{../mat/op2_tab}
  \end{tabular}
  \caption{Theorethical Current flowing through each component.}
  \label{tab:comp}
\end{table}
\par

%=========================================================================================================================0


