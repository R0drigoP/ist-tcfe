\section{Theoretical Analysis}
\label{sec:analysis}

In this section, the circuit shown earlier in Figure \ref{fig:circuit} is analyzed, using both node and mesh methods, as explained in section \ref{introduction}. It is important to notice that node 0 corresponds to ground, meaning we assign it has having null potential.
\par
%A table with all the given values can be seen in table \ref{tab:value}
**tabela com os valores

\subsection{Node method}
Using Kirchhoff Current Law (KCL) at all nodes, we end up with these equations:
\begin{equation*}
\begin{cases}
  I_d = \frac{V_1 - V_6}{R_5} + I_b \\
  I_b = \frac{V_2 - V_3}{R_2} \\
  \frac{V_2 - V_3}{R_2} = \frac{V_3 - V_4}{R_1} + \frac{V_b}{R_3} \\
  V_4 - V_5 = V_a \\
  \frac{V_6 - V_5}{R_4} = \frac{V_5 - V_7}{R_1} + \frac{V_4 - V_3}{R_1} \\ %Dizer que esta ultima parcela é Ir4->r3?
  V_6 = V_c \\
  \frac{V_5 - V_7}{R_6} = \frac{V_7}{R_7} \\
  I_b = K_bV_b = K_b(V_3 - V_6) \\
  V_c = K_cI_c = K_c \frac{V_5 - V_7}{R_6} \\
\end{cases}
\end{equation*}
Where $V_i$ represents the voltage at node $i$.
\par
Substituting variables by their numeric number given in table **, and solving this sistem of linear equations using \textit{Octave}, we end up with: **results**
\par
We can get the current flowin throuh each branch using Ohm's law
(Mostrar tabelas com as correntes ou não é necessário?)

\subsection{Mesh method}
Using Kirchhoff Voltage Law (KVL) in all meshes, we can write these equations:
\begin{equation*}
\begin{cases}
  R_1I'_A + R_3(I'_A + I'_B) + R_4(I'_A + I'_C) = V_a \\
  I'_B = I_b = K_bV_b = K_bR_3(I'_A + I'_B) \\
  R_4(I'_A+I'_C) + R_6I'_C + R_7I'_C = K_cI_c = K_cI'_C \\
  I'_D = -I_d \\
\end{cases}
\end{equation*}





The circuit consists of a single V-R-C loop where a current $i(t)$ circulates. The
voltage source $v_I(t)$ drives its input, and the output voltage $v_O(t)$ is taken from
the capacitor terminals. Applying the Kirchhoff Voltage Law (KVL), a single
equation for the single loop in the circuit can be written as

\begin{equation}
  Ri(t) + v_O(t) = v_I(t).
  \label{eq:kvl}
\end{equation}

Because $v_O$ is the voltage between capacitor C's plates, it is related to the
current $i$ by
\begin{equation}
  i(t) = C\frac{dv_O}{dt}.
\end{equation}

Hence, Equation~(\ref{eq:kvl}) can be rewritten as
\begin{equation}
  RC\frac{dv_O}{dt} + v_O(t) = v_I.
  \label{eq:kvl2}
\end{equation}

Equation is a linear differencial equation whose solution is a
superposition of a natural solution $v_{On}$ and a forced solution $v_{Of}$:

\begin{equation}
  v_O(t) = v_{On}(t) + v_{Of}(t).
  \label{eq:vo_sol}
\end{equation}

As learned in the theory classes the natural solution is of the form
\begin{equation}
  v_{On}(t) = Ae^{-\frac{t}{RC}},
  \label{eq:vo_nat}
\end{equation}
where $A$ is an integration constant.

The forced solution is of the form given in Equation and is
illustrated in Figure.

\begin{equation}
  V_{Of}(t) = |\bar{V}_{Of}| cos(\omega t + \angle \bar{V}_{Of}),
  \label{eq:vo_for}
\end{equation}

\lipsum[1-1]


%\begin{figure}[h] \centering
%\includegraphics[width=0.8\linewidth]{forced.eps}
%\caption{Forced sinusoidal response.}
%\label{fig:forced}
%\end{figure}

\section{Frequency response}

\lipsum[1-1]


